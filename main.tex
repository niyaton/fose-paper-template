% 英文で執筆する場合はクラスファイルへのオプションを[T,E]としてください.
% If you want to write your paper in English, pass to [T,E] options to document class.
\documentclass[T,J]{fose} % 「コンピュータソフトウェア」用のクラスファイルは compsoft です.
\taikai{2023} % 固定です.出版委員長が毎年変更してAuthor Kitを配布してください.

\usepackage [dvipdfmx] {graphicx}

% ユーザが定義したマクロなどはここに置く.ただし学会誌のスタイルの
% 再定義は原則として避けること.

% 以下は説明のために使用したパッケージであるため,削除可能.
\usepackage{fancyvrb}
\usepackage{xurl}
\usepackage{cite}


\begin{document}

% 論文のタイトル
\title{{\foseabbrev}論文テンプレート}
% 以下の \etitle(と\@etitle)はFOSE論文フォーマット独自のマクロです.
% FOSEに投稿した論文を発展させてコンピュータソフトウェアに投稿される場合はコメントアウトしてください.
% \setetitleは奇数ページのヘッダに表示する文字列(\etitle)を設定するためのマクロです.
% タイトルが2行に渡る場合は "\\" を 使用することで任意の位置で改行をすることができます.
\setetitle{Foundation of Software Engineering}
%\etitle{Long Long Long Long Long Long \\ Long Long Long Long Long \\ Long Long Long Long Long Long Long Long Long Long Long Long Paper Title}

% タイトル,著者などが複数行にわたり,論文冒頭の著者名が日本語アブストと重複して描画された場合に以下のコメントアウトを外してください.
%\longtitle

% 著者
% 和文論文の場合,姓と名の間には半角スペースを入れ,
% 複数の著者の間は全角スペースで区切る
%
\author{徳川 家康 源 頼朝 源 頼家
%
% ここにタイトル英訳 (英文の場合は和訳) を書く.
% 英語タイトルは論文1ページ目左下,著者らの名前・所属一覧の一番上に表示される
%
% 上記\setetitle中で改行した場合は "\etitle" を削除し,改行(\\)を入れていないタイトルを記載してください.
% \ejtitleは1ページ目左下に挿入されるタイトルとして使用されます.
% また,"\etitle"はFOSE論文フォーマット独自のマクロです.
\ejtitle{\etitle}
%
% ここに著者英文表記 および
% 所属 (和文および英文) を書く.
% 複数著者の所属はまとめてよい.
%
\shozoku{Ieyasu Tokugawa}{江戸幕府}
{Edo Bakufu}
% 複数著者の所属は以下のようにまとめてよい.
\shozoku{Yoritomo Minamoto, Yoriie Minamoto}{鎌倉幕府}
{Kamakura Bakufu}
}

%
% 和文アブストラクト
% In English paper, content of Jabstract will be ignored. 
\Jabstract{%
本稿はソフトウェア工学の基礎ワークショップのために,実践的IT教育シンポジウム rePiTの論文執筆キットおよび第29回ソフトウェア工学の基礎ワークショップの論文執筆キットを基に作成したものです.
rePiTの論文執筆キットはそもそもソフトウェア科学会の論文執筆キットを基に作成したものです,
具体的な変更箇所は\ref{sec:PaperStyle}章をご参考ください.}
%
% 英文アブストラクト(本サンプルの原論文にはなし)
\Eabstract{
This document has been prepared as a sample for FOSE(Foundation of Software Engineering) based on the Author Kit for rePiT and FOSE2022.
The author kit for rePiT was originally based on author kit of JSSST Computer Software.
The detail changes are written in Sec.\ref{sec:PaperStyle}
}
%
\maketitle \thispagestyle {empty}
\section{はじめに}
論文執筆の基本的な注意事項を以下に示します.
\begin{itemize}
\item 論文本文が和文の場合,和文・英文のいずれかでアブストラクトを
書いて下さい.両方併記することもできます.英文アブストラクトを書かない場合は
場合は{\tt eabstract}環境(\verb|\Eabstract|)を空にして使って下さい.
\item 本文が英文の場合は,クラスファイルのオプションを{\verb|[T,E]|}として下さい.
また,和文タイトル・和文著者名・和文アブストラクトを併記する必要はありません.
\item カラーの図を使うことは可能ですが,論文集はJ-STAGEへの掲載(フルカラー)だけでなく近代科学社Digitalのプリントオンデマンド書籍(白黒印刷)として印刷されることも考慮して作成してください.
白黒印刷時に図が認識可能か,文章中でフルカラー前提の図に関する特定の色を指す表現がないかなど注意してください.
\item カラーの図を使用する場合は色モードをCMYKではなくRGBで画像を作成するようにしてください.
\item 画像の解像度は300dpi以上で作成するようにしてください.
\end{itemize}
その他細かな論文執筆時の注意点については論文執筆キットで配布している{\texttt sample.pdf}の4章を参照してください.

{\textbf 謝辞}\
本フォーマットの基になったスタイルファイルを作成してくださった方々に感謝します.

%\begin{adjustvboxheight} % needed only when Appendix follows
\bibliographystyle{jssst}
\bibliography{sample}
%\end{adjustvboxheight} % needed only when Appendix follows

% 以下はbibtexを使用しない場合の例です.
% 332行目と333行目をコメントアウトしてから使用してください.
% なお,この例では年数順に文献が並んでいるので適切な並び順ではありません.
%\begin{adjustvboxheight} % needed only when Appendix follows
%\begin{thebibliography}{9}
%\bibitem{fose2021} 名倉 正剛,関澤 俊弦 編:ソフトウェア工学の基礎28,日本ソフトウェア科学会{\em FOSE2021}, 近代科学社, 2021.
%\bibitem{fose2022} 角田 雅照,まつ本 真佑 編:ソフトウェア工学の基礎29,日本ソフトウェア科学会{\em FOSE2022}, 近代科学社, 2022.
%\bibitem{fose2023} 吉田 則裕,槇原 絵里奈 編:ソフトウェア工学の基礎30,日本ソフトウェア科学会{\em FOSE2023}, 近代科学社, 2023. (to appear)
%\end{thebibliography}
%\end{adjustvboxheight} % needed only when Appendix follows

%以下は付録の例です.必要ならコメントアウトして使用してください.
%なお,その際には参考文献の前後にある adjustvboxheight 環境のコメントアウトを解除してください.
%\appendix
%\section{付録A} 
%これは付録の例です.

\end{document}

